\section{Security and Data Persistence Objectives}
\label{sec:Solution}
As mentioned above, security and availability are our two main objectives in our project. For security, we mainly focused on confidentiality, meaning that the data must be secure while transmitted, and all the access attempts to such data have to be authorized.
\subsection{Access Control}
Our database is publicly available through a REST API. For this reason, we need to ensure that our servers validate the client's credentials before executing their queries. HTTP Basic Authentication is a method for an HTTP user agent (our clients) to provide a user name and password when making an HTTP request. It is a simple and lightweight technique for enforcing access controls, and it does not require cookies, session identifiers, or login pages. In order to protect the stored client's password (in the server), we use \texttt{bcrypt}. \texttt{bcrypt} is an adaptive password-hashing function that allows increasing its iteration count to make it slower.  For this reason,  \texttt{bcrypt} is guaranteed to be resistant to brute-force search attacks even with increasing attack computation power. To implement our access control mechanism, we used the~\texttt{bcrypt}\cite{bcrypt} library, written in C++. 

\subsection{Data Encryption}
In order to guarantee the confidentiality of the data while in transit--from clients to servers, as well as from the edge servers to the master servers--we assign certificates to clients, edge servers, and master servers and encrypt messages before sending them out using HTTPS. \texttt{cpp-hpplib)~\cite{httplib} provides both HTTP and HTTPS connection in C++.

\subsection{Data Persistence}
In order to ensure that our service is available and reduce the overhead on clients, we designed our system as a multi-layer architecture, which is described in more detail in the nest section.
